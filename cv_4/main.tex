\documentclass[12pt]{article}

\input{inputs/preamble}
%%%%%%%%%%%%
%% Macros %%
%%%%%%%%%%%%

% Limit = use this in the $math$ mode 
\newcommand{\Lim}[1]{\raisebox{0.5ex}{\scalebox{0.8}{$\displaystyle \lim_{#1}\;$}}}
\newcommand*\diff{\mathop{}\!\mathrm{d}}            % \diff{x} -> dx
\newcommand*\Diff[1]{\mathop{}\!\mathrm{d^#1}}      % \Diff{3}{y} -> d^{3}y
\newcommand{\ts}{\textsuperscript}                  % shortcut for superscript in text




\includeonly{
    sections/section_1.tex,
    sections/section_2.tex,
    sections/section_3.tex,
    sections/section_4.tex,
    sections/section_5.tex,
    sections/section_6.tex,
    sections/section_7.tex,
    sections/section_8.tex,
    sections/section_9.tex,
    sections/section_10.tex,
    sections/section_11.tex,
    sections/section_12.tex,
    sections/section_13.tex,
    sections/section_14.tex,
    sections/section_15.tex,
    sections/section_16.tex,
    sections/section_17.tex,
    sections/section_18.tex
}

\title{\Huge{\textbf{Matematická analýza 1 - cvičení 4}} \\ 
\huge{Limita, posloupnosti, faktoriál, \\ metoda dvou policajtů}}
\author{Phat Tran}
\date{\today}

\begin{document}
\maketitle
\thispagestyle{empty}
\setcounter{page}{0}
\clearpage

\pagestyle{mycustomstyle} 

\section{Limita funkce}

\defi{Chování limita exponenciální funkce $f(q) \coloneq q^n$.}{
    \begin{align*}
        \lim q^n 
        \begin{cases}
            +\infty, & \text{if}\ q > 1 \\
            1, & \text{if}\ q = 1 \\
            0, & \text{if}\ q = 0 \\
            0, & \text{if}\ |q| < 1 \\
            \text{DNF}, & \text{if}\ q \le -1  
        \end{cases} \\         
    \end{align*}    
}
    
\foreach \s in {1,2,3,4,5,6,7,8,9,10,11,12,13,14,15,16,17,18} {
    \def\temp{ 
        \expandafter\input{sections/section_\s.tex}
    }\temp
}

\end{document}

