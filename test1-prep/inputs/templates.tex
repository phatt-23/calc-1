%%%%%%%%%%%%%%%%%%%%%%%%%%%%%%
%%  Quick Access Templates  %%
%%%%%%%%%%%%%%%%%%%%%%%%%%%%%%

%%% NUMBERLINE
\begin{center}
    \begin{tikzpicture}[scale=2.0]
        \draw[very thick] (1,0) -- (2,0);
        \path[draw=black, fill=black] (1,0) circle (2pt);
        \path[draw=black, fill=white, thick] (2,0.0) circle (2pt);
        \draw[latex-latex] (-3.5,0) -- (3.5,0) ;
        \foreach \x in {-3,-2,-1,0,1,2,3}
        \draw[shift={(\x,0)},color=black] (0pt,3pt) -- (0pt,-3pt);
        \foreach \x in {-3,-2,-1,0,1,2,3}
        \draw[shift={(\x,0)},color=black] (0pt,0pt) -- (0pt,-3pt) node[below] 
        {$\x$};
    \end{tikzpicture}
\end{center}

%%% Function plot
\begin{center}
    \begin{tikzpicture}
        \begin{axis}[
            axis lines=middle,
            restrict x to domain=-6:6,
            restrict y to domain=-10:20,
            xmin=-6.5,
            xmax=6.5, 
            ymin=-5.5,
            ymax=10.5, 
            xlabel=$x$,
            ylabel=$y$,
            title=The function $f(x)$ and its deriavative $f'(x)$,
            % xticklabel style={anchor=south west},
            xtick={-6, -4, -2, 1, 15/8, 3, 4, 6},
            xticklabels={$-6$, $-4$, $-2$, $1$, $\frac{15}{8}$, $3$, $4$, $6$},
            xmajorgrids=true,
        ]
        % \begin{axis}[xmin=-10, xmax=10, ymin=-10, ymax=10, axis lines = middle]
            \addplot[blue, samples=100]{(5 - 2*x)*(x^3)}
            node[right, pos=0.55]{$f(x) = (5 - 2x)3^{3}$};
            \addplot[red, samples=100]{(x^2)*(-8*x + 15)} 
            node[left, pos=0.18]{$f'(x) = x^{2}(-8x + 15)$};
        \end{axis}
    \end{tikzpicture}
\end{center}

%%% Function plot using \foreach loop

\begin{tikzpicture}
    \begin{axis}[
        axis lines=middle,
        samples=500,
        restrict x to domain=-4:4,
        restrict y to domain=-10:10,
        xlabel=$x$,
        ylabel=$y$,
        xmin=-3.5,
        xmax=3.5,
        ymin=-4.5,
        ymax=10.5,
        xmajorgrids=true,
        ymajorgrids=true,
    ]
    \foreach \COLOR/\EXPR/\LABEL/\POS/\LR in { 
        Green / { (5-(2*x))*(x^3) } / $f(x)$  / 0.6 / right, 
        Blue  / { (15-8*x)*x^2 }    / $f'(x)$ / 0.1 / left
    } 
    {
        \edef\temp{
            \noexpand\addplot[line width=2pt,color=\COLOR]{\EXPR}
            node[\LR,pos=\POS]{\LABEL};
        }
        \temp
    } 
    \end{axis}
\end{tikzpicture}

                            % \edef is expandable definition. 
                            % this will expand all the contents first.
                            % \noexpand unsures that the function \addplot
                            % is not executed until macro \temp.

%%% Function graph plotting using table and python
\begin{tikzpicture}[scale=1.5]
    \begin{axis}[
        axis lines=middle,
        xlabel=$x$,
        ylabel=$y$,
        xmin=-0.007,        % Remember to adjust the dimensions
        xmax=0.007,
        ymin=-1.7,
        ymax=1.7,
        xmajorgrids=true,
        ymajorgrids=true,
        raw gnuplot,
    ]
    \foreach \COLOR/\PATH/\LR/\POS/\LABEL in {
        Green /{tables/table_1/plot_data.txt}/ right / 0.9 / {$f(x)$}
    } {
        \edef\temp{
            \noexpand\addplot[line width=2pt, color=\COLOR] 
            table[x index=0, y index=1]{\PATH}
            node[\LR,pos=\POS]{\LABEL};
        }\temp
    }
    \end{axis}
\end{tikzpicture}  
                            % COLOR /{PATH}/ LR / POS / LABEL
                            % {PATH} the path can't have separaters

%%% Node arrow in a graph
\node (src) at (axis cs: <x>,<y>)  {\LABEL};
\node (dest) at (axis cs: <x>,<y>)  {\textbullet};
\draw[line width=1.2pt, ->, shorten>=-5pt, shorten <= -3pt]
(src)--(dest);