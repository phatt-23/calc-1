To find local extrema we need to solve for $f'(x) = 0$.
    
        \begin{align}
        f'(x) &= \left( (5-2x)x^{3} \right)' \\
        &= (5-2x)'x^{3} + (5-2x)(x^{3})' \\
        &= (-2)x^{3} + (5-2x)(3x^{2}) \\
        &= -2x^{3} + 15x^{2} - 6x^{3} \\
        &= 15x^{2} - 8x^{3} \\
        &= x^{2}(15 - 8x)
        \end{align}

        Now find $x_{1} \ldots x_{n}$ that 
        satisfy this statement $x^{2}(15 - 4x) = 0$.

        $$ x^{2}(15 - 8x) = 0 $$
        $$ (x^{2} = 0) \vee (15 - 8x = 0) 
        \Rightarrow (x^{2}(15 - 8x) = 0) $$
        % \setlength{\jot}{8pt}% tweak
        \begin{align}
            x^{2} &= 0 \\
            x_{1} &= 0 \\[10pt]
            15 - 8x &= 0 \\
            8x &= 15 \\
            x_{2} &= \frac{15}{8}
        \end{align}
        
        Now that we extracted the extrema we need to 
        decide for each wheter it is a~minimum or a~maximum.
        To do that we can use a~number line that is segmented
        by the points $x_{1}$ and $x_{2}$, and find out if 
        the function $f'(x)$ is positive or negative at each
        interval. That will helps understand how the function 
        behaves.    

        \vspace{10pt}

        \begin{center}
             \begin{tikzpicture}[scale=2.0]
                \draw[latex-latex] (-2.5,0)--(4.5,0);
                \foreach \x in {-2,-1,0,1,2,3,4}
                    \draw[shift={(\x,0)},color=white] (0pt,0pt)--(0pt,-4pt) 
                    node[right,color=black]{$\x$};
                \draw[color=white] (15/8,0)--(15/8,-2pt)
                    node[below,color=black]{$\frac{15}{8}$};
                \foreach \x in {-2,-1,0,1,15/8,2,3,4}
                    \draw[shift={(\x,0)},color=black] (0pt,2pt)--(0pt,-2pt);
                \foreach \x in {0,15/8}
                    \draw[shift={(\x,0)},color=black] (0,10pt)--(0,0);
                \path[draw=black,fill=white] (0,10pt) circle(5pt) node[]{$x_{1}$};
                \path[draw=black,fill=white] (15/8,10pt) circle(5pt) node[]{$x_{2}$};
            \end{tikzpicture}
        \end{center}
        
        \begin{center}
            For the interval $(-\infty, 0)$ we will pick $-1$ as our
            input to the function $f(x)$.
            \begin{align*}
                f'(x) &= (15 - 8x)x^{2} \\
                f'(-1) &= (15 - 8(-1))(-1)^{2} \\
                &= (15 + 8)1 
                = 24?
            \end{align*}
            We got $24$ which is positive so the function $f'(x)$ in 
            the interval $(-\infty, 0)$ is negative.

            \clearpage

            For the interval $(0, \frac{15}{8})$ we will pick $1$.
            \begin{align*}
                f'(x) &= (15 - 8x)x^{2} \\
                f'(1) &= (15 - 2(1))1^{2} 
                = 7
            \end{align*}
            The function $f'$ in $(0, \frac{15}{8})$ is positive.
        
            \vspace{0.5cm}

            For the interval $(\frac{15}{8}, +\infty)$ we choose $2$.
            \begin{align*}
                f(x) &= (15 - 8x)x^{2} \\
                f(2) &= (15 - 8(2))2^{2} 
                = -2
            \end{align*}
            The function $f'$ is negative in $(\frac{15}{8}, +\infty)$.
        \end{center}

        Now that we evaluated all the intervals 
        let's put all of our findings into a separate table.
        
        \begin{center}
            \begin{tabular}{|c|c|c|c|}
                \hline
                & $(-\infty, 0)$ & $(0, \frac{15}{8})$ & $(\frac{15}{8}, +\infty)$ \\ \hline
                $f'(x)$ & $+$ & $+$ & $-$ \\ \hline
            \end{tabular}
        \end{center}
        
        This tells us that our $x_{1}$ isn't a~local extremum because
        the intervals at either side are positive, theyx don't
        switch signs. This implies that $f'$ is decreasing, 
        touching zero and then increasing, but never switches signs.
        The $x_{2}$ on the other hand is a~local extermum, a maximum, becuase 
        the interval on left is positive and negative on the right 
        implying $f'$~is dropping down, touching zero and descreasing 
        further. The sign switch from positive to negative as they cross
        $x = \frac{15}{8}$.
        \clearpage

        \begin{center}
            \textit{The function $f$ and its fisrt derivative $f'$}
        \end{center}
        \begin{align*}
            f(x) &\coloneq (5 - 2x)x^{3} \\
            f'(x) &\coloneq (15 - 8x)x^{2}
        \end{align*}
        \begin{center}
            \begin{tikzpicture}[scale=1.5]
                \begin{axis}[
                    axis lines=middle,samples=500,
                    restrict x to domain=-4:4,
                    restrict y to domain=-10:10,
                    xlabel=$x$,ylabel=$y$,
                    xmin=-3.5,xmax=3.5,
                    ymin=-4.5,ymax=10.5,
                    xmajorgrids=true,
                    ymajorgrids=true,
                ]
                \foreach \COLOR/\EXPR/\LABEL/\POS/\LR in { 
                    Green / { (5-(2*x))*(x^3) } / $f(x)$  / 0.6 / right, 
                    Blue  / { (15-8*x)*x^2 }    / $f'(x)$ / 0.1 / left
                } 
                {
                    \edef\temp{
                        \noexpand\addplot[line width=2pt,color=\COLOR]{\EXPR}
                        node[\LR,pos=\POS]{\LABEL};
                    }
                    \temp
                } 
                \end{axis}
            \end{tikzpicture}
        \end{center}