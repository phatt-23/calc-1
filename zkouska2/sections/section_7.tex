\begin{tcolorbox}
Rozhodněte, která z následujicích tvrzení jsou pravdivá/nepravdivá.
\begin{enumerate}[
        label=(\alph*), 
        noitemsep,
        topsep=0pt,
        parsep=2pt,
        partopsep=0pt,
        labelwidth=5cm,
        align=right,
        itemindent=1.2cm
    ]
	\item Jestliže funkce $f$ není v bodě $x_0$ spojitá, pak přímka $x= x_0$ je svislou asymptotou funkce $f$.
	\item Každá periodická funkce je omezená.
	\item Je-li $f'(1) = 0$ a $f''(1) = 5$, má funkce $f$ v bodě $x = 1$ lokální extrém.
	\item Na intervalu $(0, +\infty)$ platí $\int \frac{1}{x} \diff{x} = 2021 + \ln(2022x)$.
	\item Existuje funkce $f$ spojitá na intervalu $I$, která na $I$ není omezená.
\end{enumerate}
\end{tcolorbox}

\begin{tcolorbox}[title=1. otázka]
	Ano. Body po $x$-ové ose se k tomuto bodu $x_0$ nekonečně přibližují.
\end{tcolorbox}


\begin{tcolorbox}[title=2. otázka]
	Ne. Třeba funkce $f(x) \coloneq \tan{x}$ je periodická, ale není zhora ani zdora omezená.
\end{tcolorbox}


\begin{tcolorbox}[title=3. otázka]
	Ano. Její první derivace je nulová a druhá derivace není nulová. Pokud by byla druhá derivace nulová. znamenalo by to, že je v bodě $x_0$ inflexní bod.
\end{tcolorbox}


\begin{tcolorbox}[title=4. otázka]
	\begin{align}
		F(x) &= 2021 + \ln{2022x} \\
		f(x) &= \frac{1}{x} \\
		F'(x) = (2021 + \ln(2022x))' &= \frac{1}{2022x}2022 = \frac{1}{x} = f(x) 
	\end{align}
	Platí. Absolutní hodnotu u $\ln$ můžeme vynechat protože se zabýváme jen kladnými $x$ ($\mathbb{R}_0$).
\end{tcolorbox}


\begin{tcolorbox}[title=5. otázka]
	Ano, pokud je interval $I$ otevřený, můžou hodnoty v mezních bodech nabývat $\pm\infty$, tím jsou neomezené.
\end{tcolorbox}
